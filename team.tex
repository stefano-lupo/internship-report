\chapter{Team Management and Team Process}
As with any type of team, it is critical that the team is managed appropriately. However, \textit{appropriately} doesn't imply that the team should be managed \textit{strictly}. The \team{} team are a very self-motivated team and thus required no micro management. The day to day managerial tasks fall to the tech-lead (TL). The TL is an experienced engineer who is officially responsible for the success of the team. The TL of \team{} frequently liaises with other teams and senior staff in order to maintain alignment between the goals of the company as a whole and the goals of the team. The \team{} team also recently received a project manager (PM). This role is responsible for working with customers and ensuring the team is prioritizing what will have the most impact on the customers and business as a whole. In the case of the \team{} team, neither the PM nor the TL were concerned with the day-to-day operations of a given engineer on the team. Each of the team members is sufficiently interested and invested in the success of the team that trust is placed in each of the team members to pull their weight. This management strategy aligns directly with HubSpot's ethos of \textit{"use good judgment"}. HubSpot aims to hire the type of employees that can be entirely trusted and believes in letting their employees work in whatever way they are most comfortable, which is typically the way in which they are the most productive.

However, no matter the management style, a solid \textit{team process} is paramount for the success of the team. This involves having well defined methods for managing tasks and monitoring the productivity, morale and competency of the team as a whole. Some of the critical parts of the team process of the \team{} team are outlined in the following sections.

\section{GitHub Enterprise and Issue Tracking}\label{sec:issueTracking}
HubSpot's entire engineering department runs on GitHub Enterprise \cite{githubEnterp}. This is an online platform which provides code hosting, comprehensive version control through Git \cite{git}, tracking issues and managing pull requests. 

A common method of keeping track of what needs to be done inside of a software development team is creating \textit{issues}. An issue is typically a paragraph or two long, outlining a task that the team should work towards to create, fix or improve upon something. A good issue will have context as to \textbf{why} the issue is important, \textbf{who} the issue impacts (negatively or positively) and the overall impact of the issue (for example is it critical or should other more pressing issues be prioritized). 

Internally, HubSpot is a very open company and a lot of teams work very closely together. As such it is perfectly acceptable (and encouraged) that someone from outside of the team create an issue to inform the team of something they think the team should know. Sometimes, the request contained in the issue is actually already in place - for example a request for a feature which already exists. An engineer on the team will typically respond to the issue and can close the issue once a desirable outcome has been reached. 

Critically, issues provided a comprehensive and searchable history of what the team has been working on. Issues often result in a piece of code being (re)written and a pull request submitted. GitHub offers a mechanism for cross referencing issues and pull requests so the issue which incentivized the pull request (and the pull request which resolves the issue) are forever searchable in the Git history of the project. This is extremely useful for engineers who wish to gain some broader context on a piece of the system, allowing them to read the context contained in the pull request and associated issue. 

\section{Task Management}
In order to maximize productivity and maintain a healthy workload, a system must be put in place which dictates how team members are assigned units of work. All units of work were defined in GitHub issues (see \refsec{sec:issueTracking}). The method used by the \team{} team, was to split tasks up into the following five stages:

\begin{labeling}{In-Progress}
	\item [Backlog] This contains tasks which are of low priority and can be deferred until a later date
	\item [Design] This contains tasks which require a considerable amount of planning before being undertaken. Tasks in this stage often result in discussions among the team about how best to tackle the task.
	\item [Ready] Tasks which have enough detail and context, that they are ready to be undertaken. More complex tasks that were once in the design category land here once all the corresponding details have been decided upon. These are tasks which engineers on the team should choose as a next task upon completion of a task.
	\item [In-Progress] This contains tasks which are currently being worked on (usually) by a single engineer.
	\item [Completed] Tasks which have been completed this week.
\end{labeling}

All of the tasks are managed through an online platform called Waffle \cite{waffle} which presents a \textit{"Waffle Board"} which displays the tasks in each of the above stages. The team concludes each week with a meeting in which the board is inspected and tasks are moved to new stages as appropriate. This provides a great mechanism for monitoring the team's productivity and ensuring that the team is focused on the most important tasks. 

\section{Communication}
The main method for communication among employees in HubSpot is through Slack \cite{slack}. Slack is a messaging platform which allows users to create and join channels. At HubSpot, most teams have a well known channel in which all members of the team are present in. People with questions or comments can simply join the Slack channel and send a message. Slack channels and messages are well indexed allowing users to search by keyword to find what they are looking for. This is extremely useful for looking for answers to questions that have likely been answered before, or to find the appropriate Slack channel to ask a question in.

HubSpot also has a number of channels in which engineers can ask for help. For example there is a channel specifically for Java questions where Java experts and beginners can help one and other out. These Slack channels are also very useful to follow in order to pick up random pieces of information. Often times, questions that are asked stimulate interesting conversations, allowing engineers to gain a deeper insight into what was initially asked.

\section{Proactive Operations Reviews}
Operations Reviews (Ops Reviews for short) are meetings in which the health of the system over the past time period is examined. For the \team{} team, these are conducted weekly. HubSpot has a PagerDuty service which allows alerts to be set up based on the system's current health. If one of these alerts triggers, the on-call engineer on the team receives a notification informing them of the alert. These are critical alerts that must be looked into immediately and rectified. The main purpose of the weekly ops review is to examine the alerts that triggered during the previous week. Any alerts whose root cause remains unknown at this point gets a GitHub issue created and is assigned to an engineer to investigate. Sometimes alerts can trigger prematurely, for example if a threshold value is too low and should be safely increased. In this case, an issue is created to tweak the alert to be less strict. The point of the alerts is to inform an engineer of critical problems with the system. Thus, the rate of false-positives should be kept to a minimum, reducing the overall number of alerts that trigger, while ensuring that alerts that do trigger are in fact critical to the health of the system.

This has been a key part of the \team{} team's success. There is no brushing issues with the system under the rug. Any critical issues that arise are investigated and reported on. This hugely increases the stability of the system over time. 

\section{Code Walkthroughs}
As discussed in \refsec{sec:emailSendingInfra}, the \team{} team is a relatively new team and has inherited a large portion of the code base they own. As such, engineers on the team have likely not yet encountered certain parts of the system. Each week, an engineer from the team hosts a code walkthrough. The engineer chooses a part of the system or a piece of code that they have never seen before and studies what it's purpose is and how it is implemented. The code walkthrough is a 30 minute meeting in which the engineer presents what they have learned about the part of the system they studied to the entire team. This allows other team members to gain an insight into a part of the system they have likely never seen before, without the need to spend time digging through it themselves. The engineer hosting the code walkthrough rotates in a round-robin fashion, allowing the team as a whole to gain more of an insight into a system they inherited as the weeks go by. 

\section{Critical Situation Post Mortem}
Depending on the team and product, HubSpot managers and tech leads have a set of parameters for what constitutes a critical situation (crit sit). For the \team{} team, a system issue that effects a significant portion of customers' email sending capabilities constitutes a crit sit. Should a crit sit arise, once the system is stable again, there is a well defined, formal procedure that must be followed. This procedure involves documenting the events that lead to the crit sit, the root cause of the crit sit, the remedial action and measures that should be put in place to prevent the situation from arising in the future. The team typically also has a meeting with a senior manager of the company in which the above procedure is run through.

Critically, there is never any blame assigned during a critical situation. The key reasoning behind the crit sit procedure is to identify the issue, learn from the root cause and put preventative measures in place. The crit sit procedure is a valuable tool used in HubSpot and forces teams to evaluate problems in their systems and to improve the system's stability over time.

\section{Daily Standups}
The \team{} team also has a daily standup meeting. This meeting is extremely brief but helps bring the team together at least once a day. This allows the engineers to discuss what they are currently working on, allowing team members to offer advice and have some context when the inevitable pull request is submitted. Daily standups have proved very useful as often times, more experienced team members will help identify a potential issue to watch out for when tackling a certain problem. This can help less experienced engineers avoid spending hours on an implementation only to realize a certain complexity was overlooked. 
