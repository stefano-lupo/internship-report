\section{Tasks Undertaken}
\textbf{TODO: Move task management to team info?}
asda
As with any team, a system must be put in place which dictates how team members are assigned units of work. The method used by the \team{} team, was to split tasks up into the following five stages:
\begin{description}
	\item [Backlog] This contains tasks which are of low priority and can be deferred until a later date
	\item [Design] This contains tasks which require a considerable amount of planning before being undertaken. Tasks in this stage often result in discussions amongst the team about how best to tackle the task.
	\item [Ready] Tasks which are sufficiently specified that they can be undertaken. More complex tasks that were once in the design category land here once all of the corresponding details have been decided upon. These are tasks which engineers on the team should choose as a next task upon completion of a task.
	\item [In-Progress] This contains tasks which are currently being worked on (usually) by a single engineer.
	\item [Completed] Tasks which have been completed this week.
\end{description}

All of the tasks are managed through an online platform called Waffle\cite{waffle} which presents a \textit{"Waffle Board"} which displays the tasks in each of the above stages. The team concludes each week with a meeting in which the board is inspected and tasks are moved to new stages as appropriate. This provides a great mechanism for monitoring the team's productivity and ensuring that the team is focused on the most important tasks. 

Throughout the course of the internship, a variety these tasks were undertaken and completed.  

For the sake of brevity a small subset of the most interesting of these tasks are outlined below. 