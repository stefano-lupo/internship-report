\chapter{Tasks Undertaken}
Throughout the course of the internship, a variety of tasks were undertaken and completed.  
For the sake of brevity a small subset of the most interesting of these tasks are outlined below. 

\section{Rebuilt DNS Management System}

\section{Auto DNS for dedicated IPs} \label{sec:autoDns}

\section{EmailMtaSending Kafka}

\section{CIDR Minimization Algorithm}
\textbf{MAKE SURE THIS MAKES SENSE WRT DNS, DISCUSS STRUCTURE OF SPF}\hfill\break
As discussed in \refsec{sec:DNS}, SPF records are a vital part of authorization when it comes to sending emails. SPF records are typically used to specify a list of IP addresses that a particular domain may send emails from. An important aspect of SPF records (or more specifically, the underlying TXT record) is that the length of the entire record value (which is a simple string) should be at most 255 characters as per RFC 7208 \cite{spfRFC}. Given the fact that a given HubSpot customer may potentially send email over any one of tens of HubSpot owned IPs, this can cause problems. As discussed in \refsec{sec:emailSendingInfra}, one of the upgrades customers can avail of is purchasing dedicated IP addresses, which will be used for their email traffic and theirs only. HubSpot owns a large number of IP addresses in order to facilitate this. One of the decisions that needed to be made by support staff working with customers was which IP address(es) to assign to customers. Some customers have existing IP addresses and IPs should be selected in order to minimize the length of the resulting SPF record that the customer will have. SPF records support CIDR notation (see \refsec{sec:CIDR}) of IP addresses, meaning smart IP selection can save valuable characters in a customer's SPF record. As HubSpot is moving towards automating the setting up of customer accounts with dedicated IP addresses (see \refsec{sec:autoDns}), this IP address selection needed to be automated, while still minimizing the resulting SPF records.

\subsection{CIDR Notation} \label{sec:CIDR}
CIDR (Classless Inter-Domain Routing) notation is a notation for compactly representing sets of IP addresses. This section will primarily discuss CIDR notation for version four (IPv4) IP addresses, though all of the same logic holds for version six (IPv6) IP addresses. Typically IP addresses are represented as quartet of period separated integers ranging from 0 - 255, for example, $192.168.1.1$. However, this representation is simply employed in order to make reading IP addresses easier to humans. In actuality, version four IP addresses are more simply represented as 32 bit integers. Each of the numbers in the quartet can take on one of 256 values. Thus

$$\log_2 256 = 8 bits per element in quartet$$
$$8 bits per quartet \times 4 elements in quartet = 32 bits$$

$192.168.1.1$ could be represented as a 32 bit integer by using $192$ as the most upper (most significant) 8 bits, $168$ as the next 8 bits and so on. CIDR notation contains the IP address in question, followed by a slash and a number. For example $192.168.1.2/31$. CIDR notation partitions the 32 bit representation of the IP address into two pieces - the upper bits make up the network prefix and the remaining bits are used to specify the specific host on that netowrk. The number following the slash denotes the number of bits to use for the network prefix. Thus $192.168.1.3/31$ specifies that all but the last bit should be used for the network prefix, implying that the address to reach the subnet this host is inside of is $192.168.1.2$. This is because the least significant byte of this IP address is 3 ($11_2$) and the last bit is to be zeroed, meaning the last byte of the network address is 2 ($10_2$). 

However in the context of minimizing SPF records, a simpler view of CIDR notation can be adopted. The ultimate goal is to represent a set of IP address in as few characters as possible. The set of IPs $\{192.168.1.0, 192.168.1.1\}$ can be represented using CIDR notation as $192.168.1.0/31$. Thus if a customer owns those two IP addresses, their SPF record can simply contain the CIDR notation equivalent of the two IP addresses, reducing the number of characters required by almost half. This is due to the fact that $/31$ implies that there is one bit (the last bit) which identifies the host on the subnetwork defined by $192.168.1.0$. This bit can either be a zero or a one, yielding the two possible IP addresses that were started with - $192.168.1.0$ or $192.168.1.1$.


%^ is pretty ropey, reread and make some adjustments before continuing with the alg

% CIDR notation works by specifying a bit mask that should be used for a logical $AND$ operation with the IP address. The bit masks are specified using the conventional slash notation. $/32$ implies that the bit mask contains 32 ones, $/31$ implies that the bit mask contains 31 ones and 1 zero (the zero being the least significant bit) and so on. 