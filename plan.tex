

\begin{enumerate}
	\item Introduction (5 Pages)
	\begin{enumerate}
    	\item Who are HubSpot - 2
        \item The Email Sending Infrastructure Team - 2 (team structure (what a TL is), what they all do, Hubspot small team methodology, cross team work)
        \item What the team owns (products)
        \item Goals of the team or something - 1
    \end{enumerate}
    
    \item HubSpot Methodology (10 Pages)
    \begin{enumerate}
    	\item Java MicroServices (unit, integ, accept)- 5
        \item React Native / Redux front end - 4
        \item Team Process
        \begin{enumerate}
            \item Task Management
            \item Ops Review (System Health)
            \item Code Walkthroughs
            \item Inter Team Coordination
        \end{enumerate}	
    \end{enumerate}
    
	
    \item Technologies Used (~26 pages, probably excessive)
    \begin{enumerate}
    	\item Java 8 Features (streams, CF, lambdas, Executors) - 5
        \item Kafka - 5
        \item Hadoop - 3
        \item DNS - 3
        \item SMTP - 2
        \item MySql (InnoDB) - 3
        \item HBase (Sync, Idempot, Locks) - 3
        \item ZooKeeper (+ Circus) - 2
        \item gRPC + Proto
    \end{enumerate}    
    
    \item Major Projects Undertaken (20)
    \begin{enumerate}
    	\item Rebuilt DNS management - 5
        \item Auto DNS for dedicated accounts - 5
    	\item EmailMtaSending Kafka stuff - 5
    	\item Minimize CIDR IP selection alg - go into detail about this (eg talk about SPF records in email, CIDr notation and diagram out the alg, performance etc)- 3 pages
    \end{enumerate}
    
\end{enumerate}


Useful stuff from template
Three useful online resources make \LaTeX~much better:
\begin{enumerate}
\item An excellent online \LaTeX{} environment called ``Overleaf'' is available at \url{http://www.overleaf.com} that runs in a modern web browser. It's got this template available -- search for a TCD template. Overleaf can work in conjunction with Dropbox, Google Drive and, in beta, GitHub.
\item Google Scholar, at \url{http://scholar.google.com}, provides BibTeX entries for most of the academic references it finds.
\item An indispensable and very fine introduction to using \LaTeX{} called \emph{``The not so short introduction to LATEX 2$\varepsilon$''} by \citet{oetiker2001not} is online at \url{https://doi.org/10.3929/ethz-a-004398225}. Browse it before you use \LaTeX~for the first time and  read it carefully when you get down to business.
\end{enumerate}
Other tools worth mentioning include:
\begin{itemize}
\item \texttt{Draw.io} -- an online drawing package that can output PDFs to Google Drive -- see \url{https://www.draw.io}.
\end{itemize}

\textbf{CHECK THIS - I double we need these headings..}
There are a number of chapters that you must have: an introduction; a background or literature review chapter; and a conclusion chapter. The focus of the other chapters will depend on your specific project.
